%%%%%%%%%%%%%%%%%%%%%%%%%%%%%%%%%%%%%%%%%%%%%%%%%%%%%%%%
%   |-------------------------|                        %
%   |REDES PRIVADAS VIRTUALES |                        %
%   |                         |                        %
%   | Proyecto de graduaci�n  |                        %
%   |_________________________|                        %
%                                                      %
%   Autores                                            %
%   -------                                            %
%                                                      %
% * Formoso Requena, Nicol�s Federico (CX02-0456-6)    %
%     nicolasformoso@gmail.com                         %
% * Cortez, Gustavo Maximiliano (CX01-0801-9)          %
%     cmgustavo83@gmail.com                            %
%                                                      %
%   Tutores                                            %
%   -------                                            %
%                                                      %
% * Ing. Gustavo Vanetta - vanettag@gmail.com          %
% * Lic. Miguel Bazzano - miguelbazzano@gmail.com      %
%                                                      %
%%%%%%%%%%%%%%%%%%%%%%%%%%%%%%%%%%%%%%%%%%%%%%%%%%%%%%%%

% ********* Bibliograf�as ********** %

\begin{thebibliography}{99}

\bibitem{redesprivadasvirtualesconlinux} Oleg Kolesnikov y Brian Hatch. Gu�a Avanzada Redes Privadas Virtuales con Linux. PEARSON EDUCACI�N, S.A., Madrid, 2003.

\bibitem{openbsdasavpnsolution} Alex Withers. OpenBSD as a VPN Solution. Sys Admin, the journal for UNIX and Linux systems administrators (\href{http://www.samag.com}{http://www.samag.com}).

\bibitem{revistatuxinfo} Marcelo Guazzardo. VPN, T�neles en el ciber espacio. Revista TuxInfo, A�o 1, N�mero 2, Diciembre de 2007.

\bibitem{man} Manual en OpenBSD o Linux seg�n corresponda. Sintaxis: `man \emph{comando}'.

\bibitem{ipsec} The official IPSec Howto for Linux (\href{http://www.ipsec-howto.org}{http://www.ipsec-howto.org}).

\bibitem{dh-wiki} Diffie-Hellman - Wikipedia (\href{http://es.wikipedia.org/wiki/Diffie-Hellman}{http://es.wikipedia.org/wiki/Diffie-Hellman})

\bibitem{cert-wiki} Certificados Digitales - Wikipedia (\href{http://es.wikipedia.org/wiki/Certificado_digital}{http://es.wikipedia.org/wiki/Certificado\_digital})

\bibitem{ca-wiki} Autoridad de Certificaci�n - Wikipedia (\href{http://es.wikipedia.org/wiki/Autoridad_de_certificaci�n}{http://es.wikipedia.org/wiki/Autoridad\_de\_certificaci�n})

\bibitem{x509-wiki} El est�ndar X.509 - Wikipedia (\href{http://es.wikipedia.org/wiki/X.509}{http://es.wikipedia.org/wiki/X.509})

\bibitem{wikibooks_openvpn} OpenVPN de Wikilibros (\href{http://es.wikibooks.org/wiki/OpenVPN}{http://es.wikibooks.org/wiki/OpenVPN})

\bibitem{rfc} Grupo de traducci�n al castellano de RFC (\href{http://www.rfc-es.org/}{http://www.rfc-es.org/}).

\bibitem{encarta} Microsoft MSN Encarta (\href{http://es.encarta.msn.com}{http://es.encarta.msn.com}).

\bibitem{wiki-en} Wikipedia en ingl�s (\href{http://en.wikipedia.org/}{http://en.wikipedia.org/}).

\bibitem{wiki-es} Wikipedia en castellano (\href{http://es.wikipedia.org/}{http://en.wikipedia.org/}).

\bibitem{subversion} Ben Collins-Sussman, Brian W. Fitzpatrick y Michael Pilato. Control de versiones con Subversion.

\bibitem{wikilibroslatex} Wikilibros: Manual de \LaTeX.

\bibitem{latexcientifico} Gabriel Valiente Feruglio. Composici�n de textos cient�ficos con \LaTeX. 1999.

\bibitem{latexparahumanidades} �topos. \LaTeX para Humanidades. 28 de Noviembre de 2005.

\bibitem{latexconemacs} Joaqu�n Ataz L�pez. Creaci�n de ficheros \LaTeX con GNU Emacs. 2004.

\end{thebibliography}
